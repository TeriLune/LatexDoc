Bioróżnorodność pełni fundamentalną rolę w utrzymaniu równowagi ekologicznej. Wysoki poziom zróżnicowania biologicznego zwiększa odporność ekosystemów na stresory środowiskowe, takie jak susze, choroby czy zmiany temperatury. Badania wykazały, że ekosystemy o większej liczbie gatunków szybciej regenerują się po zaburzeniach i są bardziej efektywne w obiegu materii i energii.

Różnorodność gatunkowa roślin sprzyja utrzymaniu żyzności gleby — różne gatunki wykorzystują zasoby w odmienny sposób, co zwiększa efektywność wykorzystania składników pokarmowych. W ekosystemach wodnych zróżnicowane wspólnoty planktonu i bentosu stabilizują poziom tlenu oraz poprawiają jakość wody.

Wysoka różnorodność genetyczna w populacjach roślin i zwierząt umożliwia lepsze przystosowanie się do zmieniających się warunków klimatycznych — na przykład różnorodni zapylacze są bardziej odporni na choroby i pestycydy, co stabilizuje proces zapylania.

\begin{figure}[H]
  \centering
  \includegraphics[width=0.85\textwidth]{zdjecie.jpg}
  \caption{Ilustracja: przykładowy krajobraz ilustrujący różnorodność gatunkową. \cite{projektpulsar2025}}
  \label{fig:krajobraz}
\end{figure}

\section{Usługi ekosystemowe wynikające z bioróżnorodności}

Z punktu widzenia człowieka bioróżnorodność przekłada się bezpośrednio na usługi ekosystemowe — korzyści, jakie ludzie czerpią z przyrody. Wyróżnić można cztery podstawowe kategorie:

\begin{enumerate}
  \item \textbf{Usługi zaopatrzeniowe} — dostarczanie surowców, żywności, drewna, leków i włókien naturalnych.
  \item \textbf{Usługi regulacyjne} — stabilizacja klimatu, zapylanie roślin, oczyszczanie powietrza i wody.
  \item \textbf{Usługi wspierające} — fotosynteza, obieg azotu, rozkład materii organicznej i tworzenie gleby.
  \item \textbf{Usługi kulturowe} — wartości rekreacyjne, duchowe i edukacyjne.
\end{enumerate}

Utrata bioróżnorodności obniża jakość życia (wymiar ekonomiczny i społeczny): zmniejsza dostępność zasobów, pogarsza warunki klimatyczne i zwiększa ryzyko katastrof ekologicznych.
