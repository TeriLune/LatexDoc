Do stworzenia dokumentu wykorzystano klasę \texttt{report}, ponieważ jest ona dostosowana do prac posiadających strukturę rozdziałową i stanowi standardowy wybór dla raportów oraz dokumentów akademickich średniej długości. Klasa \texttt{book} oferuje bardziej rozbudowane funkcje, jednak są one przeznaczone głównie dla publikacji książkowych i w tym przypadku nie były konieczne.

Źródła dokumentu umieszczono w publicznym repozytorium git:
\begin{center}
  \url{https://github.com/jakubzwarycz/bioroznorodnosc}
\end{center}

Repozytorium nie zawiera danych prywatnych ani plików tymczasowych — zastosowano odpowiedni plik \texttt{.gitignore}, który eliminuje m.in. pliki kompilacyjne i dane, które nie powinny być publikowane.
